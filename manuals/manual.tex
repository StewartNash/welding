\documentclass[11pt]{article}

\usepackage{amsmath}
\usepackage{amssymb}

\begin{document}

\title{Welding: Using Artifical Neural Networks}
\author{Stewart Nash}

\maketitle

\begin{abstract}
  This white paper covers the generation of artificial neural networks to set optimal welding parameters. Specifically, it explains the welding repository located at https://github.com/StewartNash/welding and covers theory in-depth at a level which cannot be accomodated by the readme on the main page.

\section{Introduction}
  Artificial neural networks can be used to model the performance of welding when trained on the results of welding.
  
\section{Optimization}

\subsection{Gradient Descent}

\subsection{Newton-Raphson Method}

The Newton-Raphson method is an iterative method for finding the roots of a differentiable function $f$, which are solutions to the equation $f(x)=0$.

For a univariate function we have the Taylor expansion
\begin{equation}
	f(x)=\sum_{k=0}^\infty{\frac{D_x^kf(x_0)}{k!}(x-x_0)^k}
\end{equation}
We take the second-order approximation around $x_0$
\begin{equation}
	f(x)=f(x_0)+f'(x_0)(x-x_0)+\frac{1}{2}(x-x_0)f''(x-x_0)
\end{equation}

\subsection{Gauss-Newton Method}

The Gauss-Newton algorithm is an extension of Newton's method of optimization. It is used to solve non-linear least squares problems, which is equivalent to minimizing a sum of squared function values.


\end{document}
